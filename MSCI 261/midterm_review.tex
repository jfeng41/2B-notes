\documentclass[11pt, oneside]{article}
\usepackage[margin=1in]{geometry}
\geometry{letterpaper}
\usepackage{amssymb}
\usepackage[fleqn]{amsmath}
\usepackage[sharp]{easylist}
\usepackage{relsize}
\usepackage{graphicx}

\pagenumbering{gobble}              % No page numbering
\setlength{\parindent}{0em}         % No paragraph indenting
\setlength{\parskip}{0.5em}         % Paragraph spacing

\newcommand*{\begineasylist}{\begin{easylist}[itemize]\ListProperties(Style*=$\bullet$\quad, Style2*=\tiny$\blacksquare$\quad, Style3*=$\circ$\quad, Style4*=$\diamond$\quad, FinalSpace=1em, Space=0em, Space*=0em)}

\newcommand*{\begineasylistnumbered}{\begin{easylist}[enumerate]\ListProperties(Numbers=a, Space=0em, Space*=0em)}

\begin{document}

\section*{MSCI 261 Midterm Review (Chpt. 2-5)}

\subsection*{Cash Flow Diagrams}
\begineasylist

\begin{figure}[!ht]
    \centering
    \includegraphics[width=0.7\textwidth]{cash_flow_diagram.png}
\end{figure}

# Cash inflows and outflows are represented by arrows
# Each ``year'' point represents the \emph{\underline{beginning}} of that year

\end{easylist}
\subsection*{Interest}
\begineasylist

# \textbf{Compound interest}: $F = P(1 + i)^N$
## $F$ = future value (value at the \underline{end of year N})
## $P$ = present value (value at the \underline{beginning of year 0})
## $i$ = interest rate (per period)
## $N$ = number of compounding periods

# \textbf{Simple interest}: $F = PN(1 + i)$

# \textbf{Nominal interest rate}: $i_s$
## ``Normal'' way of stating interest rate
## If annual nominal rate = 12\%/year, then monthly nominal rate = 1\%/month
# \textbf{Effective interest rate}: $i_e$
## ``Actual'' interest rate
# \textbf{Converting from smaller period to large period}:
## Suppose $i_s$ is stated over a small period
## Then $i_e$ over a large period, which consists of $m$ small periods, is
\[ i_e = (1 + i_s)^m - 1 \]
## i.e. effective interest is the rate such that $P(1 + i_s)^m = P(1+i_e)$

# \textbf{Converting from large period to small period:}
## If $i_s$ is given over a large period = $m$ small periods, then interest for small period is simply
\[ i = i_s/m \]

# \textbf{Converting nominal annual to effective annual rate}:
## i.e. converting $i_s$ to $i_e$ for the same large period, which consists of $m$ small compounding periods
\[ i_e = (1 + \frac{i_s}{m})^m - 1 \qquad \text{where $m$ = \# compounding periods in a year} \]
# Continuous compounding -- compounding period is infinitesimally small
\begin{align*}
i_e &= \lim_{m \rightarrow \infty} \Big(1 + \frac{i_s}{m}\Big)^m - 1\\
&= e^{i_s} -1
\end{align*}

\end{easylist}
\subsection*{Compound Interest Factors}
\begineasylist

# Compound interest factors are just notations to represent formulas used to calculate $F$ (future value), $P$ (present value), or $A$ (annuity).

# e.g. $(F/P, i, N) \rightarrow$ returns $F$, given $P$, $i$, and $N$

# \textbf{Compound amount factor} = $(F/P, i , N) = (1 + i)^N$
## Given how much a payment is worth now, how much is it worth in $N$ years?
\[ F = P(F/P, i , N) \]

# \textbf{Present worth factor} = $(P/F, i , N) = \dfrac{1}{(1 + i)^N}$
## Given how much a payment will be worth in $N$ years, how much is it worth now?
\[ P = F(P/F, i , N) \]

\begin{figure}[!ht]
    \centering
    \includegraphics[width=0.7\textwidth]{pf.png}
\end{figure}

# \textbf{Sinking fund factor} = $(A/F, i, N) = \dfrac{i}{(1 + i)^N - 1}$
## Given how much an amount should be worth in $N$ years, how much should I deposit/pay each year (i.e. \underline{annuity}) ?
\[ A = F(A/F, i, N) \]

# \textbf{Uniform series compound amount factor} = $(F/A, i, N) = \dfrac{(1 + i)^N - 1}{i}$
## If I deposit/pay $A$ each year, how much will it be worth in $N$ years?
\[ F = A(F/A, i, N) \]

\begin{figure}[!ht]
    \centering
    \includegraphics[width=0.7\textwidth]{fa.png}
\end{figure}

# \textbf{Capital recovery factor} = $(A/P, i, N) = \dfrac{i(1+i)^N}{(1+i)^N - 1}$
## Given how much a payment is worth now, how much should I deposit/pay each year in order to recover this payment in $N$ years?
\[ A = P(A/P, i, N) \]

# \textbf{Series present worth factor} = $(P/A, i, N) = \dfrac{(1+i)^N - 1}{i(1+i)^N}$
## If I despoit/pay $A$ each year for $N$ years, how much is it all worth today?
\[ P = A(P/A, i, N) \]

\begin{figure}[!ht]
    \centering
    \includegraphics[width=0.7\textwidth]{pa.png}
\end{figure}

\end{easylist}
\subsection*{Conversion Factors}
\begineasylist

# \textbf{Arithmetic gradient to annuity conversion factor} = $(A/G, i, N) = \dfrac{1}{i} - \dfrac{N}{(1 + i)^N - 1}$
## Returns an annuity value, \textbf{not} the present worth
## Annuity increases/decreases by an amount $G$ each year
### Year 1: $A = A'$
### Year 2: $A = A' + G$
### Year 3: $A = A' + 2G$
### Year $N$: $A = A' + (N-1)G$
## First find $A_{total} = A' + G(A/G, i, N)$, then $P = A_{total}(P/A, i, N)$

\begin{figure}[!ht]
    \centering
    \includegraphics[width=0.8\textwidth]{ag.png}
\end{figure}

# \textbf{Geometric gradient to present worth conversion factor} = $(P/A, g, i, N) = \dfrac{(P/A, i^o, N)}{1+g}$
## Annuity grows by a rate $g$ each year
### Year 1: $A = A'$
### Year 2: $A = A'(1 + g)$
### Year 3: $A = A'(1 + g)^2$
### Year $N$: $A = A'(1 + g)^{N-1}$
## \textbf{Growth-adjusted interest rate} = $i^o = \dfrac{1+i}{1+g} - 1$
## If $g = i > 0$, the growth rate cancels the interest rate so $i^o = 0$, and $P = \dfrac{NA}{1 + g}$

\begin{figure}[!ht]
    \centering
    \includegraphics[width=0.8\textwidth]{pag.png}
\end{figure}

\begin{figure}[!ht]
    \centering
    \includegraphics[width=0.5\textwidth]{excel_formulas.png}
\end{figure}

\end{easylist}
\newpage
\subsection*{Calculating Present and Future Worth}
\begineasylist

# Present worth can also be calculated as = sum of revenue $-$ cost in each year, divided by the discount in that year $(1 + i)^k$
## $PW = -C_{initial} + \dfrac{R_1 - C_1}{(1 + i)^1} + \dfrac{R_2 - C_2}{(1 + i)^2} + \ldots + \dfrac{R_N - C_N}{(1 + i)^N}$

# \textbf{Capital recovery formula}:
## Given initial purchase cost $P$ (year 0) and final salvage value $S$ (year $N$), what's the annual saving $A$ required to justify this purchase?
\[ A = P(A/P, i, N) - S(A/F, i, N) = (P - S)(A/P, i, N) + S \cdot i \]

# If payment period $\neq$ compound period for annuities:

## Method 1: calculate PV or FV of each annuity individually and sum
### PV of each year = $A(P/F, i, N)$
### FV of each year = $A(F/P, i, N - \text{current year})$

## Method 2: convert compounding period $\rightarrow$ payment period (i.e. find effective interest)
### $i_e = (1 + i)^m - 1$, where $m$ = the \# of compounding periods in a payment period

## Method 3: convert annuity $\rightarrow$ equivalent annual annuity (can't use this for annuities with gradients)
### i.e. an annuity payment at the end of $m$ years is the FV of $m$ years of equivalent annual annuities
### $A_{annual} = A(A/F, i, m)$, then find PV or FV over total \# of compounding years

# If $N \rightarrow \infty$:

## Present worth of a project that continues indefinitely, with \underline{infinite series of uniform cash flows} is called the \textbf{capitalized value}
\[ P = \lim_{N \rightarrow \infty} A(P/A, i, N) = \frac{A}{i}\]

\end{easylist}
\subsection*{Comparison Methods}
\begineasylist

# 



\end{easylist}

\end{document}