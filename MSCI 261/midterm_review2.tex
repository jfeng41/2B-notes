\documentclass[11pt, oneside]{article}
\usepackage[margin=1in]{geometry}
\geometry{letterpaper}
\usepackage{amssymb}
\usepackage[fleqn]{amsmath}
\usepackage[sharp]{easylist}
\usepackage{relsize}

\pagenumbering{gobble}              % No page numbering
\setlength{\parindent}{0em}         % No paragraph indenting
\setlength{\parskip}{0.5em}         % Paragraph spacing

\newcommand*{\begineasylist}{\begin{easylist}[itemize]\ListProperties(Style*=$\bullet$\quad, Style2*=\tiny$\blacksquare$\quad, Style3*=$\circ$\quad, Style4*=$\diamond$\quad, FinalSpace=1em, Space=0em, Space*=0em)}

\newcommand*{\begineasylistnumbered}{\begin{easylist}[enumerate]\ListProperties(Numbers=a, Space=0em, Space*=0em)}

\begin{document}

\section*{MSCI 261 Midterm Review 2 (Chpt. 6-9)}

\subsection*{Depreciation}
\begineasylist

# \textbf{Market value}: value at which an asset can be sold in a market (usually estimated)

# \textbf{Book value}: value calculated for accounting purposes, using a depreciation model

# \textbf{Scrap value}: value at the end of an asset's \underline{physical} life (when it's broken up for its parts)

# \textbf{Salavage value}: value at the end of an asset's \underline{useful} life (when it's sold)

# \textbf{Straight-line depreciation}: linear diminishment of book value
\[ D_{sl}(n) = \frac{P - S}{N} = \text{amount of depreciation in period $n$} \]
\[ BV_{sl}(n) = P - nD_{sl} = \text{book value at the end of period $n$} \]
## $P = $ current market value/purchase price
## $S = $ salvage value after $N$ periods
## $N = $ \# of periods in useful life

# \textbf{Declining-balance depreciation}: proportional diminishment of book value
\[ D_{db}(n) = d \cdot BV_{db}(n-1) = \text{amount of depreciation in period $n$} \]
\[ BV_{db}(n) = P(1-d)^n = \text{book value at the end of period $n$} \]
\[ d = \sqrt[N]{\frac{S}{P}} = \text{depreciation rate, given $P$, $S$, and $N$} \]

\end{easylist}
\subsection*{Financial Accounting}
\begineasylist

# \textbf{Balance sheet}: snapshot of a firm's financial position at a point in time
## (Current assets + long-term assets) = (Current liabilities + long-term liabilities) + (Owner's equity)

# \textbf{Income statement}: summary of a firm's revenues and expenses over an accounting period
## Income before taxes = Revenues $-$ expenses
## Net income = Income before taxes $-$ taxes

# Liquidity ratios: ability of a firm to meet its current liability obligations
## \textbf{Working capital} = Current assets - Current liabilities
## \textbf{Current ratio/Working capital ratio} = $\dfrac{\text{Current assets}}{\text{Current liabilities}}$
## \textbf{Acid-test ratio/quick ratio} = $\dfrac{\text{Quick assets}}{\text{Current liabilities}}$
### Quick assets = Current assets $-$ Inventories $-$ Prepaid items

# Leverage/debt-management ratios: how much a firm relies on debt for its operations
## \textbf{Equity ratio} = $\dfrac{\text{Owner's equity}}{\text{Total assets}}$

# Efficiency/asset-management ratios: how efficiently a firm uses its assets
## \textbf{Inventory-turnover ratio} = $\dfrac{\text{Sales}}{\text{Inventories}}$

# Profitability ratios: how productively a firm employs its assets to produce profit
## \textbf{Return-on-assets ratio} = $\dfrac{\text{Net income (before extraordinary items)}}{\text{Total assets}}$
## \textbf{Return-on-equity ratio} = $\dfrac{\text{Net income (before extraordinary items)}}{\text{Total equity}}$


\end{easylist}
\subsection*{Replacement Decisions}
\begineasylist

# \textbf{Equivalent Annual Cost (EAC)}



\end{easylist}
\subsection*{Taxes}
\begineasylist

# As an approximation:
## Profit$_{\text{after-tax}}$ = Profit$_{\text{before-tax}}\times (1 - t)$
## MARR$_{\text{after-tax}}$ = MARR$_{\text{before-tax}}\times (1 - t)$
## IRR$_{\text{after-tax}}$ = IRR$_{\text{before-tax}}\times (1 - t)$
## Where $t = $ corporate tax rate

# Effects of taxes on cash flows (\underline{straight-line}):
## $-$First cost$_{at}$ = $-$First cost$_{bt} + $ First cost$_{bt} \times t / N \times (P/A, i, N)$
### First cost$ \times t / N =$ annual tax savings due to the depreciation expense
## Savings$_{at}$ = Savings$_{bt} \times (1-t)$
## Salvage value$_{at}$ = Salvage value$_{bt} \times (1-t)$

# Businesses want to depreciate assets as quickly as possible = receive tax savings earlier = savings are worth more

# \textbf{Capital Cost Allowance (CCA)}: maximum amount of depreciation that a business can claim in a year
## Usually calculated as a percentage of assets: the \textbf{CCA rate}

# \textbf{Undepreciated Capital Cost (UCC)}: value of assets from which the CCA for a year is calculated
\[ UCC_{i+1} = UCC_i + \text{purchases}_i - \text{salvages}_i - CCA_i \]

# \textbf{Half-year rule}: for net purchases, half is added to the base UCC used to calculate CCA for the current year, and half is added to that of the next year
## Intended to reduce tax savings
## Without half-year rule: $UCC_n = P(1-d)^n$
## With half-year rule: $UCC_n = P(1 - \dfrac{d}{2})(1-d)^{n-1}$

# \textbf{Capital tax factor}: present worth of an asset, taking into account all future tax benefits due to depreciation
\[ CTF = 1 - \frac{td(1 + i/2)}{(i + d)(1 + i)} \]

# \textbf{Capital salvage factor}: present worth of an asset's salvage value, taking into account the ongoing effect of tax benefits
\[ CSF = 1 - \frac{td}{i + d} \]

# Effects of taxes on cash flows (\underline{declining-balance, with half-year rule}):
## First cost$_{at}$ = First cost$_{bt} \times CTF$ 
## Savings$_{at}$ = Savings$_{bt} \times (1-t)$
## Salvage value$_{at}$ = Salvage value$_{bt} \times CSF$
## e.g. finding the annual worth of a purchase with taxes, given first cost, annual revenue, and salvage value (all before taxes):
\[ AW = -P_{bt}(A/P, i, N)CTF + A_{bt}(1 - t) + S_{bt}(A/F, i, N)CSF \]

\end{easylist}
\subsection*{Inflation}
\begineasylist

\end{easylist}

\end{document}