\documentclass[11pt, oneside]{article}
\usepackage[margin=1in]{geometry}
\geometry{letterpaper}
\usepackage{amssymb}
\usepackage[fleqn]{amsmath}
\usepackage[sharp]{easylist}
\usepackage{relsize}

\pagenumbering{gobble}              % No page numbering
\setlength{\parindent}{0em}         % No paragraph indenting
\setlength{\parskip}{0.5em}         % Paragraph spacing

\newcommand*{\begineasylist}{\begin{easylist}[itemize]\ListProperties(Style*=$\bullet$\quad, Style2*=\tiny$\blacksquare$\quad, Style3*=$\circ$\quad, Style4*=$\diamond$\quad, FinalSpace=1em, Space=0em, Space*=0em)}

\newcommand*{\begineasylistnumbered}{\begin{easylist}[enumerate]\ListProperties(Numbers=a, Space=0em, Space*=0em)}

\begin{document}

\section*{CS 240 Midterm Review (Module 1--7)}

\subsection*{Asymptotic Analysis}
\begineasylist

# Problem instance (I) -- \emph{input} for the specified problem
# Problem solution -- \emph{output} for the specified problem instance
# Problem size -- Size(I) = size of instance I

# Algorithm - a step-by-step process for carrying out a series of computations
## An algorithm A solves a problem P if, for every instance I of P, A computes a valid solution for I in \underline{finite} time

# RAM model
## Assume any memory access \& primitive operation is constant time
## Assume infinite amount of memory
## Sequential operation
## Running time is determined by the \# of memory accesses \& primitive operations

# Order notations
## \underline{$f(n) \in O(g(n))$} if $\exists \ c > 0$ and $n_0 > 0$ such that $0 \leq f(n) \leq cg(n) \ \forall \ n \geq n_0$
### $f$ ``grows no faster than'' $g$
### $f$ is ``upper-bounded'' by $g$ ($\leq$)

## \underline{$f(n) \in \Omega(g(n))$} if $\exists \ c > 0$ and $n_0 > 0$ such that $0 \leq cg(n) \leq f(n) \ \forall \ n \geq n_0$
### $f$ ``grows no slower than'' $g$
### $f$ is ``lower-bounded'' by $g$ ($\geq$)

## \underline{$f(n) \in \Theta(g(n))$} if $\exists \ c_1, c_2 > 0$ and $n_0 > 0$ such that $0 \leq c_1g(n) \leq f(n) \leq c_2g(n) \ \forall \ n \geq n_0$
### $f$ and $g$ grow at the same rate

## \underline{$f(n) \in o(g(n))$} if $\forall \ c > 0, \exists \ n_0 > 0$ such that $0 \leq f(n) < cg(n) \ \forall \ n \geq n_0$
### $f$ is ``\emph{strictly} upper-bounded'' by $g$ ($<$)

## \underline{$f(n) \in \omega(g(n))$} if $\forall \ c > 0, \exists \ n_0 > 0$ such that $0 \leq cg(n) < f(n) \ \forall \ n \geq n_0$  
### $f$ is ``\emph{strictly} lower-bounded'' by $g$ ($>$)

## Suppose $L = \lim_{n \rightarrow \infty} \dfrac{f(n)}{g(n)}$
### If $L = 0$ then $f \in o(g)$
### If $0 < L < \infty$ then $f \in \Theta(g)$
### If $L = \infty$ then $f \in \omega(g)$

## If $f \in O(g)$ and $f \in \Omega(g)$, then $f \in \Theta(g)$

# Loop analysis
## Begin from the innermost nested loop; use $\sum$ for each outer loop

# Recurrence relations analysis
## e.g. mergesort:
## Step 1: split array of length $n$ into two subarrays, of lengths $\lceil \frac{n}{2} \rceil$ and $\lfloor \frac{n}{2} \rfloor$ ($T = \Theta(n)$)
## Step 2: recursively run mergesort on subarrays ($T = T(\lceil \frac{n}{2} \rceil) + T(\lfloor \frac{n}{2} \rfloor)$)
## Step 3: merge sorted subarrays into a single sorted array ($T = \Theta(n)$)
## Thus the \underline{recurrence relation} is
\begin{align*}
T(n) &= \Theta(1) &\text{ if } n = 1\\
T(n) &= T(\lceil \frac{n}{2} \rceil) + T(\lfloor \frac{n}{2} \rfloor) + \Theta(n) &\text{ if } n > 1\\
&= 2T(\frac{n}{2}) + cn\\
&= 2(2T(\frac{n}{4}) + \frac{cn}{2}) + cn\\
&= \ldots\\
&= 2^kT(\frac{n}{2^k}) + kcn &\text{where } k = \log n\\
&= nT(1) + \log n (cn)\\
&\in \Theta(n \log n)
\end{align*}

## \textbf{In general}, $\{T(n) = T(n/2) + c\} \in \Theta(n \log n)$

\end{easylist}
\subsection*{Priorty Queues and Heaps}
\begineasylist

# \textbf{Priority queue}: an \textbf{abstract data type} containing a collection of items each with a priority

# \textbf{Heap}: binary tree with 2 structures
## Structural property: all levels of filled except the lowest, which is left-justified
## Ordering property: the parent of any node has greater value than the node itself

# The height of a heap with $n$ nodes is $\Theta(\log n)$
## Since $2^k \leq n$ (\# of nodes on all levels above) and $n \leq 2^{k+1} - 1$ (\# of nodes including this level)

# \textbf{Bubble-up algorithm}: used for heap \underline{insertion}
## If node.key $>$ node.parent.key then swap
## Brings a \underline{large} value from a leaf node \underline{up}

# \textbf{Bubble-down algorithm}: used for heap \underline{deletion}
## If node.key $<$ node.largest\_child.key then swap
## Brings a \underline{small} value from the root node \underline{down}

# Heapify with bubble-up: insert each item, total runtime = $\Theta(n \log n)$

# Heapify with bubble-down (given an unordered array): since leaf nodes can't bubble-down, start bubbling down from second-last level up ($n/2$ nodes)
## Total runtime = $\Theta(n)$


\end{easylist}
\subsection*{Sorting, Selection, Randomized Algorithms}
\begineasylist

# Every problem has an intrinsic cost/problem complexity = $C(n)$
# If a problem has complexity $\Omega(C(n))$, and an algorithm has worst-case runtime $O(C(n))$, then the algorithm is \underline{optimal}

# \textbf{Selection problem}: find the $k$-th largest element within $n$ elements
## Using sorted array = $\Theta(n \log n)$
## Using heap: heapify, then remove max from heap $k$ times = $\Theta(n + k \log n)$
## Using quick-select = $\Theta(n)$

# \textbf{Quickselect}:
## Choose pivot = $\Theta(1)$
## Partition:
### Go from outermost pair inwards, swap any pairs that are in the wrong order
### i.e. $++i$ and $--j$ until $A[i] > pivot$ and $A[j] < pivot$, then swap $i$ and $j$
### Return index pivot; array is now partitioned by the pivot value
### $\Theta(n)$
## Recursively call partition on one of the two partitions, until pivot index = desired index

## Worse case: every recursive call paritions off 1 element =  $\Theta(n^2)$
## Best case: desired element is returned on first call = $\Theta(n)$
## Average case: $\sum$ all runtimes for all permutations of the array / \# of permutations ($n!$)

# \textbf{Quicksort}:
## Same as quickselect, except recurse on both partitions instead of just one
## Worse case = $\Theta(n^2)$
## Best case = average case = $\Theta(n \log n)$

# \textbf{Randomized algorithm}: algorithm whose output depends on the input as well as some random numbers
## $T(I, R) = $ runtime given input $I$ and set of random numbers $R$
## \textbf{Expected runtime} = $T^{exp} (I) = \sum_R T(I, R) \times P(R)$
## For quickselect and quicksort, randomizing the pivot makes the expected time = average time

# \textbf{Comparison model}: 
## Data can only be accessed by:
### Comparing two elements
### Moving elements around

## \textbf{Theorem}: any correct comparison-based sorting algorithm is $\Omega(n \log n)$ (at least $n \log n$)

# Non-comparison based sorts can achieve faster than $\Omega(n \log n)$

# \textbf{Countsort}: input is array of size $n$ which only contain numbers in a consecutive key set of $k$ elements
## Count the \# of occurrences of each element (i.e. a histogram)
## Calculate where each first key in key set 
## $\in \Theta(n + k)$

# \textbf{Radix sort}: represent all elements in base $r$ (radix)
## Pad with leading 0s so all elements have the same \# of digits
## Sort elements into \underline{buckets} based on their most/least significant digit
## Make subsequent passes through every digit ($r$ digits)
## $\in \Theta(nr) \in \Theta(n)$ 

# A sorting algorithm is \underline{stable} if the order of equal (tied) keys are preserved (from the original order in the input)


\end{easylist}

\end{document}