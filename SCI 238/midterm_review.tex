\documentclass[11pt, oneside]{article}
\usepackage[margin=1in]{geometry}
\geometry{letterpaper}
\usepackage{amssymb}
\usepackage[fleqn]{amsmath}
\usepackage[sharp]{easylist}
\usepackage{relsize}

\pagenumbering{gobble}              % No page numbering
\setlength{\parindent}{0em}         % No paragraph indenting
\setlength{\parskip}{0.5em}         % Paragraph spacing

\newcommand*{\begineasylist}{\begin{easylist}[itemize]\ListProperties(Style*=$\bullet$\quad, Style2*=\tiny$\blacksquare$\quad, Style3*=$\circ$\quad, Style4*=$\diamond$\quad, FinalSpace=1em, Space=0em, Space*=0em)}

\newcommand*{\begineasylistnumbered}{\begin{easylist}[enumerate]\ListProperties(Numbers=a, Space=0em, Space*=0em)}

\begin{document}

\section*{SCI 238 Midterm Review (Chpt. 1--11)}

\subsection*{The Sky}
\begineasylist

# \textbf{Celestial sphere}: coordinates and positions are relative to Earth's surface and rotation
## \textbf{Celestial equator} = Earth's equator mapped to the sky
## \textbf{Celestial poles} = Earth's poles
## \textbf{Right ascension} = longitude lines, measured w.r.t. vernal equinox
## \textbf{Declination} = latitude lines, measured w.r.t. celestial equator

# \textbf{Ecliptic} = plane of Earth's revolution around the sun

# \textbf{Vernal equinox} = point where the sun (ecliptic) crosses the celestial equator from south to north

# \textbf{Meridian} = latitudinal line directly above a given point on the surface of Earth
## \textbf{Solar noon} = when the sun is along the meridian at the current location
### a.k.a. 12:00 \textbf{apparent solar time}

# \textbf{Zenith} = directly overhead
## Sun is at the zenith when observed at the equator, during equinoxes
## Sun is at the zenith when observed at Tropic of Cancer/Capricorn, during solstices

\end{easylist}
\subsection*{Kepler's Laws}
\begineasylist

# First Law: if $a$ = semi-major axis, $e$ = eccentricity then
## Perihelion distance = $a(1 - e)$
## Aphelion distance = $a(1 + e)$

# Second Law: planets' revolution around the sun cover the same area in equal time

# Third Law: (orbital period)$^2$ = (semi-major axis)$^3$
## $P^2 [years] = a^3 [AU]$

# Newton's version of the Third Law:
\[ p^2 = \frac{4\pi^2}{G(M_1 + M_2)}a^3 \]
## Usually used to find \underline{mass of main body}, given \underline{orbital period} and \underline{orbiting distance} of orbiting body

# Distances from 2 bodies to their common centre of mass (another Newton extension) = $m_1a_1 = m_2a_2$

\end{easylist}
\subsection*{Light}
\begineasylist

# When electrons transition between energy levels in atoms, specific fixed amounts of energy are absorbed/released
## These amounts of energy correspond to specific fixed wavelengths of EM radiation

# A \underline{hot object} emits thermal radiation, with a \textbf{continuous spectrum}
# A \underline{thin gas} emits radiation at specific wavelengths depending on its composition \& temperature
## Creates an \textbf{emission line spectrum}
# A \underline{thin gas between a light source and observer} absorbs radiation at specific wavelengths depending on its composition
## Creates a \textbf{absorption line spectrum}

# First law of thermal radiation (\textbf{Stefan-Boltzmann Law}):
## Hotter objects emits more radiation per unit surface area than cooler objects
\[ F [\frac{W}{m^2}] = \sigma T^4 \]
## Or for total power over the surface area (e.g. of a star)
\[ L [W] = 4\pi r^2 \sigma T^4 \]

# Second law of thermal radiation (\textbf{Wien's Law}):
## Hotter objects emits radiation with higher average energy (shorter average wavelength) than cooler objects
\[ \lambda_{max} [nm] = \frac{2.9 \times 10^6}{T [K]} \]

# \textbf{Doppler broadening}: \underline{wider spectral lines} indicate \underline{faster rotation} of star, since light is red-/blue-shifted more by the rotation


\end{easylist}

\end{document}